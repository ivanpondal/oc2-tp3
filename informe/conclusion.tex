Una vez concluído este trabajo, terminamos con un sistema que resultó en un
juego donde pudimos aplicar muchos de los conocimientos estudiados en la
materia.

Tuvimos que configurar la segmentación, trabajar con paginación, administrar
nuestra propia memoria dinámica, desarrollar un scheduler y correr varias tareas
para así llegar al objetivo de tener un juego completamente funcional.

Es así como con este desarrollo pudimos tener una idea general de qué es lo que
debe realizar un sistema operativo y cómo se encarga de llevar a cabo cada
responsabilidad.

Como posible mejora a futuro si interesase profundizar el estudio de alguno de
los componentes construidos se podrían realizas versiones más complejas de los
mismos, como lo podría ser un administrador de memoria más inteligente o un
scheduler sin las limitaciones de tareas actual.
