\documentclass[hidelinks,a4paper,10pt, nofootinbib]{article}
\usepackage[width=15.5cm, left=3cm, top=2.5cm, right=2cm, left=2cm, height= 24.5cm]{geometry}
\usepackage[spanish]{babel}
\usepackage[utf8]{inputenc}
\usepackage[T1]{fontenc}
\usepackage{xspace}
\usepackage{xargs}
\usepackage{fancyhdr}
\usepackage{lastpage}
\usepackage{caratula}
\usepackage[bottom]{footmisc}
\usepackage{amssymb}
\usepackage{amsmath}
\usepackage{algorithm}
\usepackage[noend]{algpseudocode}
\usepackage{hyperref} % links en índice
\usepackage{tabularx} % tablas copadas

\usepackage{graphicx}
\usepackage{sidecap}
\usepackage{wrapfig}
\usepackage{caption}

% facilitates the creation of memory maps. Start address at the bottom, end address at the top.
% syntax: \memsection{end address}{start address}{height in lines}{text in box}
\newcommand{\memsection}[4]{
\bytefieldsetup{bitheight=#3\baselineskip}    % define the height of the memsection
\bitbox[]{10}{
\texttt{#1}     % print end address
\\ \vspace{#3\baselineskip} \vspace{-2\baselineskip} \vspace{-#3pt} % do some spacing
\texttt{#2} % print start address
}
\bitbox{16}{#4} % print box with caption
}


%%fancyhdr
\pagestyle{fancy}
\thispagestyle{fancy}
\addtolength{\headheight}{1pt}
\lhead{Organización del Computador II: TP3}
\rhead{$2º$ cuatrimestre de 2015}
\cfoot{\thepage\ / \pageref{LastPage}}
\renewcommand{\footrulewidth}{0.4pt}

%%caratula
\materia{Organización del Computador II}
\titulo{Trabajo Práctico III}
\subtitulo{System Programming}
\grupo{Grupo: Tú me pixeleas}
\integrante{Costa, Manuel José Joaquín}{035/14}{manuc94@hotmail.com}
\integrante{Gatti, Mathias Nicolás}{477/14}{mathigatti@gmail.com}
\integrante{Pondal, Iván}{078/14}{ivan.pondal@gmail.com}

\begin{document}

\maketitle

\tableofcontents

\newpage
\section{Introducción}
El trabajo presentado consiste en la elaboración de un sistema mínimo mediante
el cual se profundizaron los conceptos estudiados de \textit{System
Programming}. Para esto se siguieron una serie de ejercicios por los cuales en
forma gradual se fue construyendo el producto final.

De forma análoga a la resolución de los ejercicios, la estructura del informe
seguirá la paso por paso el desarrollo describiendo cada sección junto a la
explicación de las decisiones que consideramos más importantes para el correcto
funcionamiento del trabajo. A grandes rasgos, contaremos con un pasaje a modo
protegido, configuración de segmentación e interrupciones, paginación, manejo de
tareas y un scheduler.

El sistema resultante es un juego dentro del cual los jugadores deben soltar
perros que buscan huesos y los llevan a su cucha, otorgándole así puntos al
jugador correspondiente. El jugador con más huesos será el vencedor.

Para el desarrollo del mismo contamos con \textit{Bochs} que nos permitió emular
la computadora que cargaba y ejecutaba nuestro sistema, y en lo que respecta
lenguajes de programación se utilizaron \textsc{C} y \textsc{ASM} alternándolos
según la rutina que estuviéramos desarrollando.


\newpage
\section{Desarrollo}

\newpage
\section{Resultados}

\newpage
\section{Conclusiones}

\end{document}
