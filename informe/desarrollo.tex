\subsection{Interrupciones básicas}
En esta sección explicaremos como realizamos la creación de la \textit{Interrupt Descriptors Table} (IDT) y su posterior carga en el \textit{kernel}.

\subsubsection{Inicialización de la IDT}
La IDT es un arreglo de tamaño 255 de \texttt{idt\_entry}, la cual posee los campos a setear de una \textit{interrupt gate}. Como en este apartado solo nos interesa asociar las excepciones internas del procesador, únicamente generamos las primeras 20 entradas de la tabla, siguiendo la tabla 6-1 del tercer volumen del manual de Intel \footnote{Intel 64 and IA-32 Architectures Software Developer's Manual, Volume 3: System programming guide, Chapter 6}. Para inicializarla, utilizamos la función \texttt{idt\_inicializar} definida en \texttt{idt.c}, y cuya implementación se basa en llamar 20 veces al macro \texttt{IDT\_ENTRY}, definido en el mismo archivo, con los respectivos números de la excepción y el privilegio del descriptor (que en este caso querremos que sea siempre 0, pues es el \textit{kernel} quien manejará estas excepciones).

La principal tarea de esta sección es, entonces, configurar correctamente la macro \texttt{IDT\_ENTRY}. También tendremos definidas 20 rutinas de atención para cada excepción, aunque de momento lo único que harán es mover el número de la excepción a \texttt{eax} y detener la ejecución del sistema. En el futuro, la rutina se encargará de desalojar la tarea que produjo la excepción y continuar con la ejecución.
En la figura (\ref{fig:interrupt-gate}) mostramos como seteamos la entrada de la IDT correspondiente a la excepción $n$ del procesador.
\vspace{1em}

\begin{figure}[H]
\begin{center}
\begin{bytefield}[endianness=big,bitwidth=0.03125\linewidth]{32}
	\bitheader{0,4,5,7,8,12,13,14,15,16,31}\\
	\bitbox{16}{\texttt{\&\_isr}\emph{n}[31:16]} & \bitbox{1}{1} & \bitbox{2}{00} & \bitbox{1}{0} & \bitbox{1}{1} & \bitbox{3}{110} & \bitbox{3}{000} & \bitbox{5}{00000} \\
  	\bitbox{16}{Offset (parte alta)} & \bitbox{1}{P} & \bitbox{2}{DPL} & \bitbox{1}{M} & \bitbox{1}{D} & \bitbox{3}{\footnotesize Interrupt gate} & \bitbox{3}{M} & \bitbox{5}{Reserved} \\
\end{bytefield}

\begin{bytefield}[endianness=big,bitwidth=0.03125\linewidth]{32}
	\bitheader{0,16,31}\\
	\bitbox{16}{0x40} & \bitbox{16}{\texttt{\&\_isr}\emph{n}[15:0]}\\
  	\bitbox{16}{Segment Selector} & \bitbox{16}{Offset (parte baja)}\\
\end{bytefield}
\end{center}
\caption{Interrupt gate correspondiente a la excepción $n$ del procesador. \texttt{\_isr}$n$ es la rutina de atención de la interrupción $n$, mientras que \&\texttt{\_isr}$n$ es la dirección donde se encuentra alojada la misma; el bit D en 1 indica que el gate es de 32bits. La M implica que el manual pide que esos bits estén seteados de esa forma.}  
\label{fig:interrupt-gate}
\end{figure}

Es importante destacar que el selector de segmento 0x40 corresponde a un segmento de código, pues recordemos que lo que esperamos encontrar en la dirección 0x40:\texttt{\&\_isr}\emph{n} es el código de un \textit{handler} de interrupción.
Además, el segmento requiere privilegio de \textit{kernel}, lo cual es razonable pues esperamos que sea este quien resuelva las excepciones. 

\subsubsection{Carga de la IDT}
Para cargar el descriptor de la IDT (compuesto por su posición base y su tamaño) al IDTR usamos la operación \texttt{lidt} de la siguiente forma desde \texttt{kernel.asm}
\begin{lstlisting}
	lidt [IDT_DESC]
\end{lstlisting}

\subsection{Paginación básica}

Aquí contaremos los pasos necesarios para poder activar la paginación y que esta
esté configurada para realizar \textit{identity mapping}. Este término se
utiliza para describir un direccionamiento de memoria donde la dirección virtual
coincide uno a uno con la física.

Las páginas que definimos nos mapean las direcciones
\texttt{0x00000000} a \texttt{0x003FFFFF}, el directorio de páginas se encuentra
en la dirección \texttt{0x27000} y la tabla de páginas en \texttt{0x28000}.

\subsubsection{Rutinas de mapeado y desmapeado}

Escribimos dos rutinas para realizar la tarea, la de mapeado que nos permite
asociar una dirección virtual con una física, y la de desmapeado que dada una
dirección virtual la desasocia de memoria.

\subsubsection*{Mapeado}

La función \texttt{mmu\_mapear\_pagina} toma como parámetros la dirección \texttt{virtual}, el contenido del
registro \texttt{cr3}, la dirección \texttt{física} y por último los atributos
\texttt{attrs}.

\begin{enumerate}
	\item Extraemos la dirección del directorio de tablas de \texttt{cr3}.
	\begin{lstlisting}
	base_directorio_tablas = 0xFFFFF000 AND cr3
	\end{lstlisting}

	\item Calculamos el offset dentro del directorio de tablas en base a
	\texttt{virtual}.
	\begin{lstlisting}
	offset_directorio_tablas = ((0xFFC00000 AND virtual) >> 22) << 2
	\end{lstlisting}

	\item Obtenemos el Page Directory Entry correspondiente.
	\begin{lstlisting}
	pde = *(base_directorio_tablas + offset_directorio_tablas)
	\end{lstlisting}

	\item Extraemos la dirección del directorio de páginas de \texttt{pde}.
	\begin{lstlisting}
	base_directorio_paginas = 0xFFFFF000 AND pde
	\end{lstlisting}

	\item Calculamos el offset dentro del directorio de páginas en base a
	\texttt{virtual}.
	\begin{lstlisting}
	offset_directorio_paginas = ((0x003FF000 AND virtual) >> 12) << 2
	\end{lstlisting}

	\item Obtenemos el puntero al Page Table Entry correspondiente.
	\begin{lstlisting}
	ptr_pte = base_directorio_paginas + offset_directorio_paginas
	\end{lstlisting}

	\item Asignamos al Page Table Entry la dirección de la \texttt{física} y los
	atributos \texttt{attrs}.
	\begin{lstlisting}
	*ptr_pte = (0xFFFFF000 AND fisica) OR attrs
	\end{lstlisting}
\end{enumerate}

\subsubsection*{Desmapeado}

La función \texttt{mmu\_unmapear\_pagina} toma como parámetros la dirección
\texttt{virtual} y el contenido del registro \texttt{cr3}. Los primeros 6 pasos
serán idénticos a los de \texttt{mmu\_mapear\_pagina}.

\begin{enumerate}
	\setcounter{enumi}{6}
	\item Seteamos en el Page Table Entry el bit de \texttt{present} en 0.
	\begin{lstlisting}
	*ptr_pte = *ptr_pte AND 0xFFFFFFFE
	\end{lstlisting}
\end{enumerate}

\subsubsection{Generación del directorio de páginas}

Para mapear las direcciones solicitadas, desarrollamos la función
\texttt{mmu\_inicializar\_dir\_kernel} que se encarga de generar la primer
entrada en el directorio de tablas y su tabla de páginas asociada.

\begin{enumerate}
	\item Primero creamos nuestro Page Directory Entry en la dirección
		\texttt{0x27000}.
	\begin{lstlisting}
	ptr_pde = 0x27000
	*ptr_pde = 0 // Limpiamos la entrada
	*ptr_pde = *ptr_pde OR 0x3 // Asignamos atributos
	\end{lstlisting}
	El número \texttt{0x3} equivale en binario a \texttt{11}. Estos dos primeros
	bits encendidos corresponden dentro del Page Directory Entry a que el
	directorio de páginas estará presente y que además podrá ser leído y
	escrito.
	\begin{lstlisting}
	*ptr_pde = *ptr_pde OR 0x28000 // Asignamos direccion
	\end{lstlisting}

	\item Llenamos el directorio de páginas aplicándole a cada entrada los
		mismos atributos descritos en el punto anterior.
	\begin{lstlisting}
	Para i de 0 a 1024
		direccion_pagina = i << 12
		mmu_mapear_pagina(direccion_pagina, 0x27000, direccion_pagina, 0x3)
	Fin para
	\end{lstlisting}
\end{enumerate}

\subsubsection{Activación de la paginación}

Finalmente, teniendo las rutinas explicadas previamente, procedimos a activar la
paginación en el procesador. El código siguiente corresponde a una sección del
kernel.

\begin{enumerate}
	\item Llamamos a nuestra rutina de inicialización del directorio de tablas
	\begin{lstlisting}
	call mmu_inicializar_dir_kernel
	\end{lstlisting}

	\item Cargamos la dirección del directorio de tablas en el registro
		\texttt{cr3}.
	\begin{lstlisting}
	mov eax, 0x00027000
	mov cr3, eax
	\end{lstlisting}

	\item Activamos la paginación encendiendo el bit más significativo del
		registro \texttt{cr0}.
	\begin{lstlisting}
	mov eax, cr0
	or eax, 0x80000000
	mov cr0, eax
	\end{lstlisting}
\end{enumerate}
