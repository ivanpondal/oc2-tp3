El trabajo presentado consiste en la elaboración de un sistema mínimo mediante
el cual se profundizaron los conceptos estudiados de \textit{System
Programming}. Para esto se siguieron una serie de ejercicios por los cuales en
forma gradual se fue construyendo el producto final.

De forma análoga a la resolución de los ejercicios, la estructura del informe
seguirá la paso por paso el desarrollo describiendo cada sección junto a la
explicación de las decisiones que consideramos más importantes para el correcto
funcionamiento del trabajo. A grandes rasgos, contaremos con un pasaje a modo
protegido, configuración de segmentación e interrupciones, paginación, manejo de
tareas y un scheduler.

El sistema resultante es un juego dentro del cual los jugadores deben soltar
perros que buscan huesos y los llevan a su cucha, otorgándole así puntos al
jugador correspondiente. El jugador con más huesos será el vencedor.

Para el desarrollo del mismo contamos con \textit{Bochs} que nos permitió emular
la computadora que cargaba y ejecutaba nuestro sistema, y en lo que respecta
lenguajes de programación se utilizaron \textsc{C} y \textsc{ASM} alternándolos
según la rutina que estuviéramos desarrollando.
